%! TeX program = lualatex
\documentclass[a4paper]{article} 

% packages
\usepackage{microtype}      % Slightly tweak font spacing for aesthetics
\usepackage[english]{babel} % Language hyphenation and typographical rules
\usepackage[final, colorlinks = false, urlcolor = cyan]{hyperref} 
\usepackage{changepage}     % adjust margins on the fly

\usepackage{fontspec}
\setmainfont{EB Garamond}
\setmonofont[Scale=MatchLowercase]{Deja Vu Sans Mono}

\usepackage{minted}
\usemintedstyle{algol_nu}
\usepackage{xcolor}

\usepackage{pgfplots}
\pgfplotsset{width=\textwidth,compat=1.9}

\usepackage{caption}
\newenvironment{code}{\captionsetup{type=listing}}{}
\captionsetup[listing]{skip=0pt}
\setlength{\abovecaptionskip}{5pt}
\setlength{\belowcaptionskip}{5pt}

\usepackage[yyyymmdd]{datetime}
\renewcommand{\dateseparator}{--}

\usepackage{titlesec}
% \titleformat{\section}{\LARGE\bfseries}{}{}{}[\titlerule]
% \titleformat{\subsection}{\Large\bfseries}{}{0em}{}
% \titlespacing{\subsection}{0em}{-0.7em}{0em}
%
% \titleformat{\subsubsection}{\large\bfseries}{}{0em}{$\bullet$ }
% \titlespacing{\subsubsection}{1em}{-0.7em}{0em}

% margins
\addtolength{\hoffset}{-2.25cm}
\addtolength{\textwidth}{4.5cm}
\addtolength{\voffset}{-3.25cm}
\addtolength{\textheight}{5cm}
\setlength{\parskip}{0pt}
\setlength{\parindent}{0in}
% \setcounter{secnumdepth}{0}

\begin{document}
\hrule \medskip
\begin{minipage}{0.295\textwidth} 
    \raggedright
    \footnotesize 
    Name: Andrew Hayes \\
    E-mail: \href{mailto://a.hayes18@universityofgalway.ie}{\texttt{a.hayes18@universityofgalway.ie}}  \hfill\\   
    ID: 21321503 \hfill
\end{minipage}
\begin{minipage}{0.4\textwidth} 
    \centering 
    \vspace{0.4em}
    \Large 
    \textbf{CT3112} \\ 
\end{minipage}
\begin{minipage}{0.295\textwidth} 
    \raggedleft
    \today
\end{minipage}
\medskip\hrule 
\begin{center}
    \normalsize
    Assignment 01: Presentation Skills
\end{center}
\hrule

\section{Preparation \& Planning}
The first thing that I considered when I was planning this presentation was my role as a presenter: what information do
I want to convey, and how best to convey it? 
After I had chosen my topic (why Vim is my favourite text editor), I asked myself the 6 question: Why, Who, Where, When,
What, \& How.
Most of the time I spent on this presentation was on the preparation \& planning.
I tried to follow the 5:1 ratio: since my presentation was to be 20 minutes long, I prepared for just over 100 minutes.

\begin{itemize}
    \item   ``\textbf{Why} am I making this presentation? What is it's purpose?''
            I decided that the purpose of my presentation would be to inform someone about what Vim is, how it works,
            and whether or not Vim might be the right text editor for them.
            I wanted to inform \& empower the listener to facilitate them beginning their own journey using Vim as their
            text editor.
    \item   ``\textbf{Who} is this presentation for?''
            I decided that I wanted my presentation to be accessible to all people, not just those with a Computer
            Science background. 
            This informed a lot of my decisions throughout the presentation as to how much detail to go into: I wanted
            to make sure that anybody could listen to and follow along with my presentation if they were interested in
            it.
    \item   ``\textbf{Where} will this presentation be delivered?''
            The constraints of this assignment dictated that the presentation be submitted as a pre-recorded video,
            meaning that I could not have any audience participation or Q\&A sessions.
            Because there could be no audience interaction in my presentation, I felt that it was important I made sure
            my presentation wasn't just a boring ``death by PowerPoint'' slide after slide of plain text, so I decided
            to regularly switch from the presentation slides to a practical demonstration, to keep it engaging \&
            interesting for the listener.
    \item   ``\textbf{When} will this presentation be delivered?'' 
            Unfortunately, due to the pre-recorded nature of the presentation, I have no control and no knowledge over
            when the audience would be seeing my presentation.
            If I did have control over this, I would pick the morning-time, when the audience is most awake, alert, \&
            interested so the risk of me boring them or the information going over their heads is at its lowest, but
            since I didn't have control, I had to prepare for the worst-case scenario.
            I had to assume the audience would be bored, hungry, \& tired when viewing my presentation, so I aimed to
            make it as engaging, concise, and interesting as I possibly could.
    \item   ``\textbf{What} do I want to present?''
            Once I knew what my purpose, audience, and constraints were, I needed to consider what information I
            actually wanted to present to the audience.
            I decided that I wanted a short, pithy, \& concise overview of the subject, with minimal digressions and
            plenty of practical examples \& demonstrations, as I felt that this fit my purpose the best.
    \item   ``\textbf{How} do I want to present my information?
            As I was limited to a pre-recorded video, I unfortunately was not able to walk around a stage to make my
            presentation more visually interesting or something similar.
            I had to focus my effort on the appearance of my presentation slides and the manner in which I conveyed
            information.
            I decided to keep my slides as minimal as possible, only inserting images where I thought it was important
            and avoiding putting huge walls of texts on the slides, just bullet points.
            I decided not to use many images as I wanted to keep the slide deck short, and I felt that me switching to
            demonstrations regularly would make up for the lack of colour in my slides, instead allowing them to be
            streamlined and filled with only the most important \& relevant information.
            Since there was no camera on me, just the audio from my laptop microphone, I tried to speak clearly \&
            loudly, and tried my best to avoid lip-smacking or saying ``uhhh'' excessively to allow for a more pleasant
            listening experience.
\end{itemize}

\subsection{Structuring}
I wrote my presentation in the following order:
\begin{enumerate}
    \item   Purpose \& Objective.
    \item   Middle: main content.
    \item   End: Summary, conclusions, \& recommendations.
    \item   Beginning: Introduction/Opening.
\end{enumerate}

After I had my purpose \& objective determined, I set to writing the ``middle part'', or the slides that would
constitute the bulk of the information in the presentation.
I also loosely scripted the demonstrations that I wanted to perform, and when I wanted to perform them, which I had as
cue cards off-screen.
\\\\
Then, once I had my content completed, I wrote a conclusion / summary of all the information that I had presented, and
wrote a slide where I recommended whether or not the listener should use Vim, and some alternatives that they might be
interested in.
I decided to move the slide in which I explain how to get set up with Vim at the very end of the presentation, just
before the summary, as I felt that it was important to tell the audience whether or not they should be interested in
setting up Vim for their own use before telling them how to set it up.
I believe that this ensured that I maximised the audience engagement and retention, and maintained chronological
ordering.
\\\\
I also considered the barriers that might exist to effective communication: language, cultural, \& technological.
In particular, I tried to minimise the technological barriers, as I was speaking about a program that was typically only
used by programmers \& computer scientists, so I wanted to make sure the information was presented in a way that was
accessible to everyone.
\\\\
Finally, once I had all other content written, I wrote the introductory slides to the topic, which was much easier now
that I knew exactly what content I was introducing.
In the past, I wrote the introductions first, and then had to go back and edit them as my content changed, so writing
the introduction last was a real game-changer for me, and something that I will continue to do into the future.
\\\\
My goal was to make the content logical, well-structured, \& easy to follow, starting with simple introductory concepts,
and slowly progressing to the complex in an incremental manner.
I tried to give each slide a clear \& concise heading to provide structure to the presentation and to make each point
flow smoothly \& logically to the next point

\section{Presenting}
When I was presenting my slides, I wanted to make sure that I was providing more information than just reading off the
slides, so I added extra information \& demonstrations as I saw fit as I went along, in a loosely-scripted manner.
I tried to ensure that my information \& speech was clear, concise, and as brief as possible without being too short to
convey the information clearly.
\end{document}
