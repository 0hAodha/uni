%! TeX program = lualatex
\documentclass[a4paper,11pt]{article} 
% packages
\usepackage{censor}
\StopCensoring
\usepackage{fontspec}
\setmainfont{EB Garamond}
% for tironian et fallback
% % \directlua{luaotfload.add_fallback
% % ("emojifallback",
% %      {"Noto Serif:mode=harf"}
% % )}
% % \setmainfont{EB Garamond}[RawFeature={fallback=emojifallback}]

\setmonofont[Scale=MatchLowercase]{Deja Vu Sans Mono}
\usepackage[a4paper,left=2cm,right=2cm,top=\dimexpr15mm+1.5\baselineskip,bottom=2cm]{geometry}
\setlength{\parindent}{0pt}

\usepackage{fancyhdr}       % Headers and footers 
\fancyhead[R]{\normalfont \leftmark}
\fancyhead[L]{}
\pagestyle{fancy}

\usepackage{microtype}      % Slightly tweak font spacing for aesthetics
\usepackage[english]{babel} % Language hyphenation and typographical rules
\usepackage{xcolor}
\definecolor{linkblue}{RGB}{0, 64, 128}
\usepackage[final, colorlinks = false, urlcolor = linkblue]{hyperref} 
% \newcommand{\secref}[1]{\textbf{§~\nameref{#1}}}
\newcommand{\secref}[1]{\textbf{§\ref{#1}~\nameref{#1}}}

\usepackage{changepage}     % adjust margins on the fly

\usepackage{minted}
\usemintedstyle{algol_nu}

\usepackage{pgfplots}
\pgfplotsset{width=\textwidth,compat=1.9}

\usepackage{caption}
\newenvironment{code}{\captionsetup{type=listing}}{}
\captionsetup[listing]{skip=0pt}
\setlength{\abovecaptionskip}{5pt}
\setlength{\belowcaptionskip}{5pt}

\usepackage[yyyymmdd]{datetime}
\renewcommand{\dateseparator}{--}

\usepackage{enumitem}

\usepackage{titlesec}

\author{Andrew Hayes}

\begin{document}
\begin{titlepage}
    \begin{center}
        \hrule
        \vspace*{0.6cm}
        \censor{\huge \textbf{CT4100}}
        \vspace*{0.6cm}
        \hrule
        \LARGE
        \vspace{0.5cm}
            Information Retrieval
        \vspace{0.5cm}
        \hrule

        \vfill
        \vfill

        \hrule
        \begin{minipage}{0.495\textwidth} 
            \vspace{0.4em}
            \raggedright
            \normalsize 
            Name: Andrew Hayes \\
            E-mail: \href{mailto://a.hayes18@universityofgalway.ie}{\texttt{a.hayes18@universityofgalway.ie}}  \hfill\\   
            Student ID: 21321503 \hfill
        \end{minipage}
        \begin{minipage}{0.495\textwidth} 
            \raggedleft
            \vspace*{0.8cm}
            \Large
            \today
            \vspace*{0.6cm}
        \end{minipage}
        \medskip\hrule 
    \end{center}
\end{titlepage}

\pagenumbering{roman}
\newpage
\tableofcontents
\newpage
\setcounter{page}{1}
\pagenumbering{arabic}

\section{Introduction}
\subsection{Lecturer Contact Details}
\begin{itemize}
    \item   Colm O'Riordan.
    \item   \href{mailto://colm.oriordan@universityofgalway.ie}{\texttt{colm.oriordan@universityofgalway.ie}}.
\end{itemize}

\subsection{Motivations}
\begin{itemize}
    \item   To study/analyse techniques to deal suitably with the large amounts (\& types) of information.
    \item   Emphasis on research \& practice in Information Retrieval.
\end{itemize}

\subsection{Related Fields}
\begin{itemize}
    \item   Artificial Intelligence.
    \item   Database \& Information Systems.
    \item   Algorithms.
    \item   Human-Computer Interaction.
\end{itemize}

\subsection{Recommended Texts}
\begin{itemize}
    \item   \textit{Modern Information Retrieval} -- Riberio-Neto \& Baeza-Yates (several copies in library).
    \item   \textit{Information Retrieval} -- Grossman.
    \item   \textit{Introduction to Information Retrieval} -- Christopher Manning.
    \item   Extra resources such as research papers will be recommended as extra reading.
\end{itemize}

\subsection{Grading}
\begin{itemize}
    \item   Exam: 70\%.
    \item   Assignment 1: 30\%.
    \item   Assignment 2: 30\%.
\end{itemize}

There will be exercise sheets posted for most lecturers; these are not mandatory and are intended as a study aid.

\subsection{Introduction to Information Retrieval}
\textbf{Information Retrieval (IR)} deals with identifying relevant information based on users' information needs, e.g.
web search engines, digital libraries, \& recommender systems. 
It is finding material (usually documents) of an unstructured nature that satisfies an information need within large
collections (usually stored on computers).












\end{document}
