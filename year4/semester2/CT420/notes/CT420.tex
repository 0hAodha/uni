% ! TeX program = lualatex
\documentclass[a4paper,11pt]{article} 
% packages
\usepackage{censor}
\StopCensoring
\usepackage{fontspec}
\setmainfont{EB Garamond}
% for tironian et fallback
% % \directlua{luaotfload.add_fallback
% % ("emojifallback",
% %      {"Noto Serif:mode=harf"}
% % )}
% % \setmainfont{EB Garamond}[RawFeature={fallback=emojifallback}]

\setmonofont[Scale=MatchLowercase]{Deja Vu Sans Mono}
\usepackage[a4paper,left=2cm,right=2cm,top=\dimexpr15mm+1.5\baselineskip,bottom=2cm]{geometry}
\setlength{\parindent}{0pt}

\usepackage{fancyhdr}       % Headers and footers 
\fancyhead[R]{\normalfont \leftmark}
\fancyhead[L]{}
\pagestyle{fancy}

\usepackage{microtype}      % Slightly tweak font spacing for aesthetics
\usepackage[english]{babel} % Language hyphenation and typographical rules
\usepackage{xcolor}
\definecolor{linkblue}{RGB}{0, 64, 128}
\usepackage[final, colorlinks = false, urlcolor = linkblue]{hyperref} 
% \newcommand{\secref}[1]{\textbf{§~\nameref{#1}}}
\newcommand{\secref}[1]{\textbf{§\ref{#1}~\nameref{#1}}}

\usepackage{changepage}     % adjust margins on the fly

\usepackage{minted}
\usemintedstyle{algol_nu}

\usepackage{pgfplots}
\pgfplotsset{width=\textwidth,compat=1.9}

\usepackage{caption}
\newenvironment{code}{\captionsetup{type=listing}}{}
\captionsetup[listing]{skip=0pt}
\setlength{\abovecaptionskip}{5pt}
\setlength{\belowcaptionskip}{5pt}

\usepackage[yyyymmdd]{datetime}
\renewcommand{\dateseparator}{--}

\usepackage{enumitem}

\usepackage{titlesec}

\author{Andrew Hayes}

\begin{document}
\begin{titlepage}
    \begin{center}
        \hrule
        \vspace*{0.6cm}
        \Huge \textsc{ct420}
        \vspace*{0.6cm}
        \hrule
        \LARGE
       \vspace{0.5cm}
       Real-Time Systems
       \vspace{0.5cm}
       \hrule

       \vfill

       \hrule
        \begin{minipage}{0.495\textwidth} 
            \vspace{0.4em}
            \raggedright
            \normalsize 
            \begin{tabular}{@{}l l}
                Name: & Andrew Hayes \\
                Student ID: & 21321503 \\
                E-mail: & \href{mailto://a.hayes18@universityofgalway.ie}{a.hayes18@universityofgalway.ie} \\
            \end{tabular}
        \end{minipage}
        \begin{minipage}{0.495\textwidth} 
            \raggedleft
            \vspace*{0.8cm}
            \Large
            \today
            \vspace*{0.6cm}
        \end{minipage}
        \medskip\hrule 
    \end{center}
\end{titlepage}

\pagenumbering{roman}
\newpage
\tableofcontents
\newpage
\setcounter{page}{1}
\pagenumbering{arabic}

\section{Introduction}
\subsection{Lecturer Contact Information}
\begin{itemize}
    \item   Name: Dr. Michael Schukat.
    \item   E-mail: \href{mailto://michael.schukat@universityofgalway.ie}{michael.schukat@universityofgalway.ie}.
    \item   Office: CSB-3002.
\end{itemize}

\begin{itemize}
    \item   Name: Dr. Jawad Manzoor.
    \item   E-mail: \href{jawad.manzoor@universityofgalway.ie}{jawad.manzoor@universityofgalway.ie}.
    \item   Office: CSB-3012.
\end{itemize}

\subsection{Assessment}
\begin{itemize}
    \item   2 hours of face-to-face \& virtual labs per week from Week 03.
    \item   30\% Continuous Assessment:
            \begin{itemize}
                \item   2 assignments, 10\% each.
                \item   2 in-class quizzes between Week 07 \& Week 12, worth 5\%.
            \end{itemize}
\end{itemize}

\subsection{Introduction to Real-Time Systems}
A system is said to be \textbf{real-time} if the total correctness of an operation depends not only upon its logical correctness but also upon the time in which it is performed.
Contrast functional requirements (logical correctness) versus non-functional requirements (time constraints).
There are two main categorisation factors:
\begin{itemize}
    \item   \textbf{Criticality:}
            \begin{itemize}
                \item   \textbf{Hard RTS:} deadlines (responsiveness) is critical.
                        Failure to meet these have severe to catastrophic consequences (e.g., injury, damage, death).
                \item   \textbf{Soft RTS:} deadlines are less critical, in many cases significant tolerance can be permitted.
            \end{itemize}

    \item   \textbf{Speed}
            \begin{itemize}
                \item   \textbf{Fast RTS:} responses in microseconds to hundreds of microseconds.
                \item   \textbf{Slow RTS:} responses in the range of seconds to days.
            \end{itemize}
\end{itemize}

A \textbf{safety-critical system (SCS)} or life-critical system is a system whose failure or malfunction may result in death or serious injury to people, loss of equipment / property or severe damage, \& environmental harm.

\section{The Essence of Time: From Measurement to Navigation \& Beyond}
\textbf{Time} is the continued sequence of existence \& events that occurs in an apparently irreversible succession from past, through the present, into the future.
Methods of temporal measurement, or chronometry, take two distinct forms:
\begin{itemize}
    \item   The \textbf{calendar}, a mathematical tool for organising intervals of term;
    \item   The \textbf{clock}, a physical mechanism that counts the passage of time.
\end{itemize}

Global (maritime) exploration requires exact maritime navigation, i.e., longitude \& latitude calculation.
\textbf{Latitude} (north-south) orientation is straightforward; \textbf{longitude} (east-west orientation) requires a robust (maritime) clock.
\\\\
\textbf{Ground-based navigation systems} like LORAN (LOng RAnge Navigation) were developed in the 1940s and were in use until recently, and required fixed terrestrial longwave radio transmitters, and receivers on-board of ships \& planes.
They are also referred to as hyperbolic navigation or multilateration.
The principles of ground-based navigation systems is as follows:
\begin{enumerate}
    \item   A \textbf{master} with a known location broadcasts a radio pulse.
    \item   Multiple \textbf{slave} stations with a known distance from the master send their own pulse, upon receiving the master pulse.
    \item   A \textbf{receiver} receives master \& slave pulses and measures the delay between them.
    \item   This allows the receiver to deduce the distance to each of the stations, providing a fix.
\end{enumerate}






\end{document}
